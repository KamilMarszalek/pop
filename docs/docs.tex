\documentclass[12pt, a4paper]{article}
\usepackage[T1]{fontenc}
\usepackage[utf8]{inputenc}
\usepackage{amsmath}
\usepackage{geometry}

\geometry{margin=0.75in}
\linespread{1.25}

\begin{document}

\title{%
	[POP] Dokumentacja wstępna projektu \\
	\Large Zadanie 13 \textendash{} karty na prostokątnej planszy o wymiarach $4 \times n$.
}
\author{
	Michał Szwejk \\ 331445 \and
	Kamil Marszałek \\ 331401
}
\date{}
\maketitle

\section{Polecenie}
W każdej komórce planszy prostokątnej o rozmiarze $4 \times n$ wpisano liczbę całkowitą $z_{ij}$.
Masz do dyspozycji $m$ kart, które musisz rozmieścić na planszy.
Poprawny rozkład kart zakłada, że żadna para kart nie może zajmować komórek sąsiadujących w pionie lub poziomie.
Twoim zadaniem jest znalezienie takiego rozkładu kart na planszy, aby suma liczb zapisanych w komórkach planszy była jak największa.
Nie musisz wykorzystywać wszystkich kart.

\section{Reprezentacja rozwiązania}
Rozwiązanie jest reprezentowane jako macierz zmiennych binarnych $x_{ij} \in \{0, 1\}$,
gdzie $i$ i $j$ odpowiadają wierszowi i kolumnie na planszy.
Podejście to można dodatkowo uprościć wykorzystując wektory mask bitowych \textendash{}
każdej kolumnie odpowiada liczba całkowita dodatnia, której reprezentacja binarna odwzorowuje przyjęte wartości.

\section{Programowanie dynamiczne}

Zagadnienie można rozwiązać wykorzystując programowanie dynamiczne.
Jako podproblem definiujemy maksymalizację sumy wartości obecnej kolumny i wartości zakumulowanej wynikającej z rozwiązania poprzednich podproblemów.
Dobierając maksymalne lokalne rozwiązanie należy uwzględnić ograniczenie sąsiedztwa (analizujemy dwie kolejne maski, ich iloczyn bitowy musi być równy $0$) oraz ograniczenie liczby kart (łączna liczba użytych kart nie może przekraczać $m$).

Formalnie, niech $DP[i][p][k]$ oznacza maksymalną sumę wartości, jaką można uzyskać
rozpatrując kolumny od $i$ do końca planszy, przy założeniu, że w poprzedniej kolumnie użyto maski $p$,
a do dyspozycji pozostało $k$ kart. Wówczas:
\[
DP[i][p][k] =
\begin{cases}
0, & \text{jeśli } i = n \text{ lub } k = 0, \\[6pt]
\displaystyle
\max_{\substack{m \in M \\ m \,\&\, p = 0 \\ |m| \le k}}
\big( W[i][m] + DP[i+1][m][k - |m|] \big),
& \text{w przeciwnym razie.}
\end{cases}
\]
Gdzie $M$ oznacza zbiór wszystkich dopuszczalnych masek kolumn (bez sąsiadujących jedynek),
$|m|$ liczbę kart wybranych w masce $m$, a $W[i][m]$ sumę wartości pól zakrytych przez maskę $m$ w kolumnie $i$.

Rozwiązanie końcowe stanowi wartość $DP[0][0][m]$, czyli maksymalny możliwy zysk dla całej planszy.

\section{Strategia zachłanna}
Z planszy wybieramy kolejno komórki, dla których opisująca ją wartość liczbowa $z_{ij}$ jest największa.
Zaznaczamy je (dołączamy do rozwiązania) i usuwamy ich sasiądów w pionie i poziomie tak długo, aż ograniczenie na liczbę kart $m$ przestanie być spełnione.
Po znalezieniu wstępnego rozwiązania strategią zachłanną dodatkowo je ulepszamy naprawiając lokalnie regiony o wymiarach $k \times 4$ wykorzystując programowanie dynamiczne.
Takie podejście pozwoli nam naprawić miejsca, w których suma wartości sąsiadów danego pola jest większa niż ono same (mimo iż pojedynczo ma większą wartość).
Regiony wybierane są losowo, a ich rozmiar i liczba wszystkich lokalnych poprawek są parametrami algorytmu.


\section{Algorytm A*}

Algorytm A* łączy podejście programowania dynamicznego z przeszukiwaniem kierowanym heurystyką. 
Każdy stan opisuje częściowe rozwiązanie (aktualną kolumnę, maskę bitową i liczbę użytych kart), 
a jego ocena wyrażona jest wzorem:
\[
    f(s) = g(s) + h(s),
\]
gdzie $g(s)$ oznacza uzyskany dotąd zysk, a $h(s)$ szacuje maksymalny możliwy zysk w dalszej części planszy.

Heurystyka $h(s)$ obliczana jest za pomocą \emph{blokowego programowania dynamicznego} — plansza dzielona jest na bloki (np. po 10 kolumn), 
dla których lokalnie wyznaczane są maksymalne wartości zgodne z ograniczeniami sąsiedztwa i liczby kart. 
Poza bieżącym blokiem wartości heurystyki są rozszerzane (\emph{propagowane}) o globalną sumę największych dodatnich elementów planszy, 
co gwarantuje, że $h(s)$ jest dopuszczalna i monotoniczna. 
Dzięki temu A* analizuje znacznie mniej stanów niż pełne DP, zachowując poprawność i optymalność wyniku.

\section{Planowane eksperymenty}
W eksperymentach porównamy jakość i czas działania zaproponowanych algorytmów na zestawie instancji o różnych rozmiarach planszy i liczbie kart. 
\end{document}
